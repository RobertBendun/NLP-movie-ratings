\documentclass{article}
\usepackage{polski}
\usepackage{listings}
\usepackage{pgfplots}
\pgfplotsset{compat=1.18}
\usepgfplotslibrary{statistics}

\lstset{
	tabsize=2,
	basicstyle=\ttfamily,
  columns=fullflexible,
  keepspaces=true,
	frame=tb,
	%numbers=left
}

\begin{document}

\section{Tytuły jako wyłączny wyznacznik oceny filmu}

\lstinputlisting[language=SQL,caption=Kwerenda wyznaczająca tytuły filmów oraz ich oceny,label=titles-query]{./queries/titles.sql}

\begin{lstlisting}[language=bash,caption=Przygotowanie danych dla modelu i ewaluacji,label=query1]
bin/db-tool vw -db db -query queries/titles.sql -out titles.txt
bin/pslit 3 titles.train.txt 1 titles.test.txt < titles.txt
\end{lstlisting}

\begin{lstlisting}[language=bash,caption=Wielkości zbiorów]
wc -l titles.{train,test}.txt
  678826 titles.train.txt
  225428 titles.test.txt
  904254 total
\end{lstlisting}

\begin{lstlisting}[language=bash, caption=Generowanie modelu]
vw -d titles.train.txt -f titles.vw
\end{lstlisting}

\begin{lstlisting}[language=bash, caption=Ewaluacja modelu]
vw -d titles.test.txt -i titles.vw -p titles.predictions.txt
\end{lstlisting}
\end{document}
